%Capítulo 4

\section{Presentación de los resultados}

    \textbf{\textit{Descripción}}
    
    Es una argumentación fundamentada con claridad y rigor científico, donde se dan a conocer los resultados encontrados, ofreciendo respuesta a cada una de las preguntas u objetivos de la investigación \cite{knuthwebsite}.
    
    \textbf{\textit{Criterios de calidad}}
    
    Las conclusiones deben de responder a la pregunta: \textbf{¿Qué resultados se encontraron y cuáles son sus beneficios?}
    Las conclusiones deben estar redactadas con: rigor lógico, claridad y concisión de estilo, originalidad, precisión, amplitud, compatibilidad con la ética, significancia y pertinencia.
    
    Es siguiente es un ejemplo de un ecuación. Llamada Tabla \ref{tab:corr}
    
    \begin{table}[h!]
        \centering
        \begin{tabular}[c]{clccc} \hline
            \# & Corrección                 & Hird          & Nuestro                   & Factor \\  
            \hline
            1  & Frecuencia. ($Hz$)         & $1$           & $340$                     & $340$ \\
            2  & Duración ($s$)             & $1$           & $60$                      & $1/60$ \\
            3  & Área detección ($cm^2$)    & $25$          & $2.5\times 10^{-6}$       & $10^{-5}$ \\
            4  & Mancha focal ($cm^2$)      & $1$           & $1$                       & $1$ \\
            5  & Distancia ($cm$)           & $1$           & $1$                       & $1$ \\
            6  & Fuentes X-Ray              & $1$           & $1$                       & $1$ \\ \hline
        \end{tabular}
        \caption{La tabla es un ejemplo}
        \label{tab:corr}
    \end{table}

\section{Conclusiones y las recomendaciones}

    \textbf{\textit{Descripción}}
    
    Es la argumentación lógica y fundamentada, probando o desaprobando la hipótesis, y dando a conocer los beneficios o alcances de la investigación
    
    \textbf{\textit{Criterios de calidad}}
    
    Las conclusiones deben de responder a la pregunta: \textbf{¿Qué resultados se encontraron y cuáles son sus beneficios?}
    Las conclusiones deben estar redactadas con: rigor lógico, claridad y concisión de estilo, originalidad, precisión, amplitud, compatibilidad con la ética, significancia y pertinencia.
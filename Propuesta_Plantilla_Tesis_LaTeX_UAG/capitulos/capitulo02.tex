%Capítulo 2

\section{Marco histórico y contextual}

    Existe un sistema en la actualidad para registrar los trabajos de tesis de los alumnos, sin embargo, este sistema fue desarrollado 
    
    "La investigación en la UAG es una actividad que contribuye a dar coherencia a sus otras funciones sustantivas y tiene por objeto promover, alentar y apoyar a los esfuerzos que lleven a generar nuevo conocimiento en las áreas académicas que integran a la institución, contribuir a la solución de los problemas de la sociedad a nivel regional y nacional, mejorar la calidad del proceso enseñanza-aprendizaje para nuestros estudiantes, vinculando pertinentemente las actividades de investigación de los profesores en conformidad con la Misión, Valores y Fines de la Universidad Autónoma de Guadalajara.
    La investigación deberá ser realizada en todos los niveles educativos que ofrece la Institución y estará focalizada en proponer soluciones a los problemas planteados por los sectores productivos, social y de servicios tanto del sector público como del privado. Es por ello y siguiendo este propósito se pretende que con el Sistema de información de la Coordinación de Investigación y Desarrollo Tecnológico se tenga un registro de toda la investigación que se crean en nuestra Universidad, desde Tesis de cualquier Licenciatura o grado académico, hasta artículos, proyectos de intervención o bien cualquier tipo de investigación que sea relevante sin importar la etapa escolar en la que se esté."\cite{UAG}
    
    
    
    
    Reseña histórica que permita identificar o describir el contexto donde se desarrollará o se desarrolló el estudio.
    Caracteriza el contexto institucional donde se realizará la investigación. No se limita únicamente a un lugar geográfico, sino que hace una ampliación descriptiva del mismo.
    Ubica la investigación en una época y tiempo determinados, y sus circunstancias particulares.
    Debe presentar una síntesis de lo que ya se encontró sobre el tema que se desea investigar (fuentes, referentes, teorías, investigaciones).
    
    \textbf{\textit{Criterios de calidad}}
    
    La página actual del sistema de investigadores de la Universidad Autónoma de Guadalajara, utiliza un diseño anticuado; antes del 2018, en el cual la mayoría de los elementos web son estáticos y, no se adaptan con fluidez a las resoluciones actuales en el año 2022, además, para incluir nuevas funcionalidades al sistema de investigadores se necesita modificar una gran cantidad de elementos web o, casi rediseñarlos, ya que la arquitectura utilizada de Modelo Vista Controlador(MVC) simple no permite su modificación de manera sencilla.
    
    La propuesta del equipo de trabajo liderado por la Doctora Lina Maria Aguilar Lobo consiste en volver a desarrollar el sistema de investigadores, pero con una nueva arquitectura y elementos que faciliten la implementación de nuevas funcionalidades. La arquitectura propuesta se compondrá de los siguientes elementos:
    
    \begin{itemize}
        \item Utilizar el diseño de Modelo Vista Controlador como base.
        \item Separar el desarrollo en dos partes: frontend y backend.
        \item Crear una base de datos relacional normalizada.
        \item Apoyarse de una arquitectura de micro servicios para proveer de información al frontend.
        \item Interfaz gráfica de usuario responsiva.
    \end{itemize}
    
    El marco teórico debe responder a la pregunta: \textbf{¿En qué está sustentada la investigación?}
    Debe presentar el sustento teórico (teorías), filosófico, pedagógico, psicológico, ético, etc. que respalda científicamente la investigación.
    Debe presentarse el escenario de argumentaciones en el cual se sitúan los investigadores, de donde se debe justificar, manifestando la opinión y punto de vista de la base de donde se va a partir, según sus fundamentos teóricos y concepción filosófica.
    
    Es siguiente es un ejemplo de un lista de elementos.
    
    \begin{itemize}
        \item La geometría del sistema,
        \item Los materiales involucrados,
        \item Las partículas fundamentales de interés,
        \item La generación de eventos primarios,
        \item El seguimiento de partículas a través de materiales y campos electromagnéticos,
        \item Los procesos físicos que rigen las interacciones de las partículas,
    \end{itemize}

\section{Marco referencial}

    \textbf{\textit{Descripción}}
    
    Reseña de las investigaciones anteriores o actuales que apoyan nuestro estudio
    
    \textbf{\textit{Criterios de calidad}}
    
    El marco teórico debe responder a la pregunta: \textbf{¿En qué está sustentada la investigación?}
    Debe presentar el sustento teórico (teorías), filosófico, pedagógico, psicológico, ético, etc. que respalda científicamente la investigación.
    Debe presentarse el escenario de argumentaciones en el cual se sitúan los investigadores, de donde se debe justificar, manifestando la opinión y punto de vista de la base de donde se va a partir, según sus fundamentos teóricos y concepción filosófica.

\section{Marco legal}

    \textbf{\textit{Descripción}}
    
    Fundamentos legales del tema de estudio. (Opcional, si se considera necesarios para el estudio)
    
    \textbf{\textit{Criterios de calidad}}
    
    El marco teórico debe responder a la pregunta: \textbf{¿En qué está sustentada la investigación?}
    Debe presentar el sustento teórico (teorías), filosófico, pedagógico, psicológico, ético, etc. que respalda científicamente la investigación.
    Debe presentarse el escenario de argumentaciones en el cual se sitúan los investigadores, de donde se debe justificar, manifestando la opinión y punto de vista de la base de donde se va a partir, según sus fundamentos teóricos y concepción filosófica.

\section{Marco Teórico}

    \textbf{\textit{Descripción}}
    
    Es el respaldo organizado en argumentos teóricos y referenciales que se le da al problema de investigación. Es la evaluación, presentación y pertinencia de enfoques y resultados de teorías e investigaciones en diversas áreas del conocimiento, los cuales han abordado directa o indirectamente, una problemática similar a la del proyecto actual.
    
    \textbf{\textit{Criterios de calidad}}
    
    El marco teórico debe responder a la pregunta: \textbf{¿En qué está sustentada la investigación?}
    Debe presentar el sustento teórico (teorías), filosófico, pedagógico, psicológico, ético, etc. que respalda científicamente la investigación.
    Debe presentarse el escenario de argumentaciones en el cual se sitúan los investigadores, de donde se debe justificar, manifestando la opinión y punto de vista de la base de donde se va a partir, según sus fundamentos teóricos y concepción filosófica.

\section{Hipótesis}

    \textbf{\textit{Descripción}}
    
    La hipótesis es una proposición formulada y estructurada de tal forma que trata de responder a una inquietud o a un problema.
    Se plantea con el fin de explicar o conocer hechos o fenómenos que caracterizan o identifican al objeto de estudio, interrelacionando una o más variables mediante una proposición lógica.
    Es una repuesta tentativa al problema de investigación.
    Puede ser enunciada en forma afirmativa, negativa o interrogativa.
    
    \textbf{\textit{Criterios de calidad}}
    
    Debe :
    Siempre de estar de acuerdo con el problema y los objetivos de investigación.
    Admitir verificación y validación estadística.
    Ser comprobable o empíricamente demostrable.
    Estar en armonía con el marco teórico y con otras hipótesis del campo de investigación.
    Cumplir con el principio de ”parsimonia”, es decir, entre dos hipótesis igualmente probables, debe de elegirse la más sencilla.
    Ser precisa, específica y expresarse con simplicidad lógica. Además de descriptiva, debe de intentar una explicación del fenómeno.
    Expresarse en forma cuantitativa, o ser susceptible de cuantificación.
    Ser generalizable.
    Ser factibles de probarse.
    No contener implicaciones morales.

\section{Definición de términos y conceptos básicos}

    \textbf{\textit{Descripción}}
    
    Define y relaciona en orden alfabético, los términos usados en la definición del problema. Cumplen con la función de:
    Establecer una mejor comunicación entre investigadores y usuarios de las investigaciones.
    Unificar criterios entre investigadores.
    Mejorar la validación de la investigación.
    Delimitar con precisión los alcances de la investigación.
    Ayudar a definir la postura del investigador.
    
    \textbf{\textit{Criterios de calidad}}
    
    La definición de términos y conceptos debe elaborarse con:\\
    \textbf{Precisión}: Los términos y conceptos relevantes deben de estar claramente determinables sin omisión de su esencia.\\
    \textbf{Complitud}: Debe asegurarse que cada uno de los términos o conceptos relevantes esté abarcado en la investigación.\\
    \textbf{Consistencia}: Las definiciones y los conceptos empleados en la investigación deberán de emplearse de igual forma en tiempo, modo y forma por todos los que participan en la investigación.\\
    \textbf{No repetición}: No pueden emplearse como parte de una definición, términos que deban ser definidos.\\
    \textbf{Compatibilidad}: Los términos y conceptos deben ser compatibles con los términos y conceptos usados normalmente por la ciencia en la cual se realiza la investigación y deben de ser, en lo posible, los utilizados por los usuarios de la investigación.\\
    \textbf{Extensión}: Los términos y conceptos no deberán ser tan extensos que lo abarque todo, pero deberán ser comprobables.\\
    \textbf{Sencillez de redacción}.\\
    \textbf{Claridad} de conceptos sin rebuscamientos lingüísticos.\\
    \textbf{Congruencia}: La definición de términos y conceptos debe ser congruente con la filosofía del investigador y el marco teórico.